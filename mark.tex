
\documentclass[11pt]{article}
  	\usepackage{ucs} 
	\usepackage[utf8x]{inputenc} % Включаем поддержку UTF8  
	\usepackage[russian]{babel}  % Включаем пакет для поддержки русского языка 
	\usepackage {mathtext}
	\usepackage{mathrsfs, amsmath, amssymb}
	\usepackage{graphicx}
	\usepackage{listings}
	\usepackage{hyperref}
	\usepackage{revsymb}
	\usepackage{listings}
\lstset{language=[90]Fortran,
  basicstyle=\ttfamily,
  keywordstyle=\color{red},
  commentstyle=\color{green},
  morecomment=[l]{!\ }% Comment only with space after !
}
	\hypersetup{
    colorlinks=true,
    linkcolor=blue,
    filecolor=magenta,      
    urlcolor=cyan,
	}
	\urlstyle{same}
	\DeclareGraphicsExtensions{.pdf,.png,.jpg,.jpeg}

	\graphicspath{{pictures/}}
    \title{\textbf{ Запутанность в марковской диссипативной динамике \\ -- \\ 
	Entanglement in Markovian Dissipative Dynamics}}
    \author{И.А.Юхновский}
    \date{ноябрь 2020}
    
\begin{document}

\maketitle
\thispagestyle{empty}
\section*{Аннотация}
В статье рассматриваются невзаимодействующие двухуровневые системы, погруженные в общий термостат, которые становятся взаимно запутанными, развиваясь согласно марковской, полностью положительной редуцированной динамике. Цель данной работы - развитие марковской динамики запутанных систем.


\section*{Abstract}
The article deals with non-interacting two-level systems immersed in a common thermostat, which become mutually entangled, developing according to Markov, completely positive reduced dynamics. The aim of this work is to develop the Markov dynamics of entangled systems.

\tableofcontents{}

\section{Введение}
Роль квантовой запутанности имеет первостепенное значение.
важность в квантовой теории информации и вычислений. В последние годы большое количество исследований было посвящено изучению того, как запутать две системы посредством прямого взаимодействия между ними (см., Например, [1–5]). В таком контексте наличие окружающей среды, например, типичного шумного резервуара или термостата, обычно считается противодействующим созданию запутанности из-за его эффектов декогерентизации и улучшения смешивания. Тем не менее, термостат может также обеспечивать косвенное взаимодействие между полностью изолированными подсистемами и, таким образом, средством их перепутывания. Действительно, это было явно показано в простой, точно решаемой модели [6]. Там корреляции между двумя подсистемами устанавливаются во время переходной фазы, когда сокращённая динамика подсистем содержит эффекты памяти. Вместо этого в этом письме мы изучаем возможность того, что запутанность создаётся термостатом во время марковского режима посредством чисто шумного механизма. 

\section{Запутанность}
Мы рассматриваем две невзаимодействующие двухуровневые системы, слабо связанные с общим термостатом. Затем мы начнём с полного гамильтониана вида:

\begin{equation}
H_{tot}=H_0^{(1)}+H_0^{(2)}+H_B+H_{int}
\label{eq_1}
\end{equation}

,где $H_0^{(1)},H_0^{(2)},H_B$ управляют динамикой двух подсистем и термостата в отсутствие друг друга; член взаимодействия связывает каждую подсистему независимо с термостатом, и его можно принять в виде

\begin{equation}
H_{int}=\sum\limits_{\alpha=1}^3(\sigma_\alpha \otimes 1) \otimes V_\alpha + \sum\limits_{\alpha=4}^6(1 \otimes \sigma_{\alpha-3}) \otimes V_{\alpha}
\label{eq_2}
\end{equation}

,где $\sigma_1, \sigma_2, \sigma_3$ - матрица Паули.

Обратите внимание, что мы позволяем подсистемам взаимодействовать с термостатом через различные операторы V, при этом исключено прямое взаимодействие между собой.

В пределе слабой связи ~\cite{b7,b8,b9,b10,b11,b12} приведённая динамика двух двухуровневых систем принимает марковский вид. Предполагая факторизованное начальное состояние $\rho \otimes \rho_B$, где $\rho$ - состояние двух подсистем и $\rho_B$ - состояние равновесия термостата, $\rho$ эволюционирует во времени в соответствии с квантовой динамической полугруппой полностью положительных отображений с генератором формы Косаковского-Линдблада:

\begin{equation}
\partial_t\rho(t) = -i[H,\rho(t)]+L[\rho(t)]
\label{eq_3}
\end{equation}
 
Унитарный член - это коммутатор с эффективным гамильтонианом $H=H^{(1)}+H^{(2)+H^{(12)}}$, состоящий из отдельных частей системы, включая Лэмб сдвиг, вызванный ванной

\begin{equation*}
H^{(1)}=\sum\limits_{i=1}^3H_i^{(1)}(\sigma_i \otimes 1)
\end{equation*}

\begin{equation}
H^{(2)}=\sum\limits_{i=1}^3H_i^{(2)}(1 \otimes \sigma_i)
\label{eq_4}
\end{equation}

плюс, возможно, двухсистемный терморегулятор, порождаемый ванной

\begin{equation}
H^{(12)}=\sum\limits_{ij=1}^3H_{ij}^{(12)}(\sigma_i \otimes \sigma_j)
\label{eq_5}
\end{equation}

Диссипативный вклад $L[\rho(t)]$ равен:

\begin{equation}
L[\rho]=\sum\limits_{\alpha,\beta=1}^6 \mathscr{D}_{\alpha,\beta}[\mathscr{F}_\alpha \rho \mathscr{F}_\beta -
\frac{1}{2}\{ \mathscr{F}_\beta \mathscr{F}_\alpha, \rho \}
]
\label{eq_6}
\end{equation}

с $\mathscr{F}_\alpha = \sigma_{\alpha} \otimes 1$ для $\alpha = 1,2,3, \mathscr{F}_\alpha = 1 \otimes \sigma_{\alpha-3}$ для $\alpha=4,5,6$, и $\mathscr{D} = \mathscr{D}^†$ положительная матрица $6 \times 6$, которая гарантирует полную положительность эволюции.

Записав
\begin{equation}
\mathscr{D} = (
\begin{array}{cc}
	A & B \\
	B^† & C\\
\end{array}
)
\label{eq_7}
\end{equation}

с $6 \times 6$ матрицами  $A=A^†$, $C=C^†$ и $B, L[\rho]$ предполагает форму, более поддающуюся физической интерпретации:

\begin{equation}
\begin{gathered}
L[\rho]=\sum\limits_{i,j=1}^3(A_{ij}[(\sigma_i \otimes 1) \rho (\sigma_j \otimes 1)-\frac{1}{2}\{ (\sigma_j\sigma_i \otimes 1), \rho\}]) \\
+ C_{ij}[(1 \otimes \sigma_i) \rho (1 \otimes \sigma_j) -\frac{1}{2}\{(1 \otimes \sigma_j \sigma_i),\rho\} \\
+ B_{ij}[(\sigma_i \otimes 1)\rho (1 \otimes \sigma_j)-\frac{1}{2}\{ (\sigma_i \otimes \sigma_j), \rho\} ] \\
+ B_{ij}^{*}[(1 \otimes \sigma_j) \rho (\sigma_i \otimes 1) - \frac{1}{2}\{(\sigma_i \otimes \sigma_j, \rho) \} 
])
\end{gathered}
\label{eq_8}
\end{equation}

Генератор такой формы применялся в квантовой оптике для описания явления коллективной резонансной флуоресценции (например, ~\cite{b13}). 
В приведённом выше выражении первые два вклада являются диссипативными членами, которые влияют на первую, соответственно, вторую систему в отсутствие другого. Напротив, последние две части представляют способ, которым шум может коррелировать две подсистемы; этот эффект присутствует, только если матрица B отличается от нуля.

\subsection{Замечание 1}
Из простого вывода основных уравнений Маркова ~\cite{b7,b8} известно, что гамильтоновы члены ~\ref{eq_4} и ~\ref{eq_5} и элементы матрицы $\mathscr{D}$ в ~\ref{eq_6} содержат интегралы от двух точечных корреляционных функций во времени операторов ванны: $Tr[\rho_BV_\alpha V_{\beta}(t)]$. В частности, матрицы $[H_{ij}^{(12)}]$ в ~\ref{eq_5} и $[B_{ij}]$ в ~\ref{eq_8} не обращаются в нуль только в том случае, если состояние ванны $\rho_B$ коррелирует операторы ванны $V_\alpha$, относящиеся к разным подсистемам, т.е. есть, если ожидания $Tr[\rho_BV_\alpha V_\beta(t)]$ отличны от нуля при $1 \le \alpha \le 3$ и $4 \le \beta \le 6$. Только в этом случае возможно запутывание под действием ванны. В самом деле, если $H^{12}=0$ и $B=0$, две подсистемы развиваются независимо, и первоначально разделимые состояния могут стать более смешанными, но определённо не запутанными. Чтобы проверить, запутывается ли уменьшенная матрица плотности $\rho$ двух систем во время $t$ из-за эволюции во времени генерируется уравнением. ~\ref{eq_3} можно использовать критерий частичного транспонирования ~\cite{14,15}: если $\rho(t)$, действующий с частичным транспонированием по отношению к одной из двух подсистем, имеет отрицательные собственные значения, то он запутан; в четырехмерном случае, который мы изучаем, также верно и обратное, а именно, если $\rho(t)$ запутано, то частичное транспонирование вызывает появление отрицательных собственных значений. С физической точки зрения ванна не способна создавать запутывание тогда и только тогда, когда частичная перестановка сохраняет положительность состояния $\rho(t)$ на все времена.

\subsection{Замечание 2}
Строго говоря, этот критерий позволяет исследовать возможность создания запутанности, начиная с разделимых начальных состояний. Когда начальное состояние уже запутано, критерий частичной перестановки не может решить вопрос; в таких случаях анализ запутывающей способности ванны может быть решен только путем изучения того, как меры запутывания меняются во времени в условиях диссипативной приведенной динамики. Эта проблема требует отдельного рассмотрения и здесь не рассматривается.

Поэтому мы принимаем разделяемые состояния в качестве начальных состояний: как мы увидим, это на самом деле не ограничение для целей обсуждения возможности создания запутанности, вызванной ванной. Далее, мы можем ограничить наше исследование чистыми состояниями; действительно, если ванна не может запутать их, она определенно не запутает их смеси. В связи с этим будем рассматривать начальные состояния вида

\begin{equation}
\rho(0)=|a_1\rangle \langle a_1| \otimes | b_1\rangle \langle b_1|
\label{eq_9}
\end{equation}

,где $\{ |a_i\rangle \}, \{ |b_i\rangle \}, i=1,2$  являются ортонормированными базисами в двумерных гильбертовых пространствах двух подсистем.

Для определённости оперируем частичным транспонированием второго множителя по базису $\{ |b_1\rangle , | b_2 \rangle \}$

Можно действовать с частичным транспонированием по обе стороны уравнения ~\ref{eq_3} и преобразовать результат как

\begin{equation}
\partial_t \tilde \rho (t) = -i[\tilde H, \tilde \rho(t)]+\tilde L [\tilde \rho (t)]
\label{eq_10}
\end{equation}

,где $\tilde \rho (t)$ обозначает частично транспонированную матрицу $\rho (t)$, $\tilde H$ новый гамильтониан, в который вносят вклад как унитарный, так и диссипативный член в ~\ref{eq_3} вносят вклад

\begin{equation}
\tilde H = \sum\limits_{i=1}^3 H_i^{(1)}(\sigma_i \otimes 1) + \sum\limits_{ij=1}^3Im(B \cdot S)_{ij}(\sigma_i \otimes \sigma_j)
\label{eq_12}
\end{equation}

,где $S$ диагональная $3 \times 3$ матрица задана $S=diag(-1, 1, -1)$
Дополнительный кусок $\tilde{L}[\cdot]$ имеет вид ~\ref{eq_6}, но с новой матрицей $\mathscr{D} \rightarrow S \cdot \tilde{\mathscr{D}} \cdot S$, где

\begin{equation}
\tilde{\mathscr{D}} = \left(
\begin{array}{cc}
A & Re(B) + iH^{(12)}\\
Re(B^T)-iH^{(12)T} & C^{T}\\
\end{array}
\right)
\label{eq_12}
\end{equation}

\begin{equation}
S = \left(
\begin{array}{cc}
1_3 & 0 \\
0 & S \\
\end{array}
\right)
\label{eq_13}
\end{equation}

а верхний индекс $T$ обозначает полное транспонирование, а $H^{(12)}$ - матрица коэффициентов в ~\ref{eq_5}.

\subsection{Замечание 3}
Хотя $\tilde \rho (t)$ эволюционирует согласно главному уравнению формально в форме Косаковски-Линдблада, новая матрица коэффициентов $\tilde{D}$ не обязательно должна быть положительной.
Как следствие, эволюция во времени, порождённая ~\ref{eq_10}, может быть ни полностью положительной, ни положительной, и, следовательно, может не сохранять положительность начального состояния $\tilde{\rho}(0) \equiv \rho(0)$.

Обратите внимание, что и гамильтониан, и диссипативный
условия исходного главного уравнения ~\ref{eq_3} вносят вклад в часть $\tilde{L}[\cdot]$ в ~\ref{eq_10} , единственный член в ~\ref{eq_10}, который может давать отрицательные собственные значения. В частности, это делает более прозрачным физический механизм, согласно которому прямая гамильтонова связь $H^{(12)}$ между двумя системами может вызывать запутанность: на $\tilde{\rho}(t),H^{(12)}$ «действует» как диссипативный вклад, который в целом не сохраняет положительность. Сила сцепления чисто гамильтоновых взаимодействий широко изучалась в недавней литературе ~\cite{b1,b2,b3,b4,b5}. Вместо этого в дальнейшем мы сосредоточим наше внимание на том, может ли запутывание быть произведено чисто диссипативным действием термостата; в дальнейшем мы не будем учитывать вклад матрицы $H^{(12)}$ в $\tilde{\mathscr{D}}$. Другими словами, мы будем учитывать только термостаты, для которых индуцированная двухсистемная гамильтонова связь в ~\ref{eq_5} исчезающе мала.

\subsection{Замечание 4}
Если $\tilde{\mathscr{D}}$ положительно, то эволюция во времени, порождённая ~\ref{eq_10}, полностью положительна; следовательно, $\tilde{\rho}(t)$ всегда положительный, и запутывание не создаётся. Экземпляры термостатов, для которых это происходит, легко могут быть
при условии:

\begin{itemize} 
	\item (i) $B=0$: в таком случае $\tilde{\mathscr{D}}$ положительно, поскольку таковы $A$ и $C^T$ в силу положительности $\tilde{\mathscr{D}}$; это соответствует термостату,  который не коррелирует динамически с двумя подсистемы;
	\item (ii) $Re(B)=0$: как прежде, $\tilde{\mathscr{D}}$ блочно-диагональный и, следовательно, положительный;
	\item (iii) $Im(B)=0$ и $C^T=C$ или $A^T=A$: в первом случае $\tilde{\mathscr{D}} = \mathscr{D}$, а во втором: $\tilde{\mathscr{D}} = \mathscr{D}^T $;
	\item (iv) $A^T=A$ и $C^T=C$: в этом случае, $\tilde{\mathscr{D}} = (\mathscr{D} + \mathscr{D}^T)/2 $
\end{itemize} 

В последних трёх случаях, несмотря на тот факт, что две подсистемы теперь динамически коррелируются с помощью термостата, эффекта недостаточно для образования сцепления. Кроме того, обратите внимание, что перепутывание не создается также в термостатах, для которых соответствующая матрица $\mathscr{D}$ коэффициентов может быть записана как выпуклая комбинация матриц, удовлетворяющих предыдущим условиям.

Чтобы проверить наличие отрицательных собственных значений в $\tilde{\rho}(t)$, вместо изучения полного уравнения ~\ref{eq_10} находим удобным изучать величину:

\begin{equation}
\mathscr{E}=\langle \psi | \tilde{\rho}(t)|\psi \rangle
\label{eq_14}
\end{equation}

, где $\psi$ - любой четырехмерный вектор. Предположим, что исходное сепарабельное состояние $\tilde{\rho}$ действительно развило отрицательное собственное значение в момент времени $t$, но не раньше. Тогда существует векторное состояние $\ \psi \rangle$ и время $t^{*}<t$ такие, что $\mathscr{E}(t^{*})=0, \mathscr{E}(t^{})>0$ для $t<t^{*}$ и E $\mathscr{E}<0$ для $t> t^{*}$. Таким образом, знак создания запутанности может быть задан отрицательной первой производной $\mathscr{E}(t)$ в $t=t^{*}$. Более того, по предположению состояние $\rho(t^{*})$ сепарабельно. Без ограничения общности можно положить $t^{*}=0$ и, как уже отмечалось, ограничить внимание факторизованными чистыми начальными состояниями

Другими словами, две подсистемы, изначально подготовленные в состоянии $\rho(0)=\tilde{\rho}(0)$, как в ~\ref{eq_9}, будут запутаны зашумленной динамикой, вызванной их независимым взаимодействием с термостатом, если ~\ref{eq_1} $\mathscr{E}(0)=0$ и ~\ref{eq_2} $\partial_t \mathscr{E}(0) < 0$, для подходящего вектора $|\psi \rangle$,

\begin{equation}
|\psi \rangle = \sum\limits_{i,j=1}^2 \psi_{ij}|a_i \rangle \otimes |b_i \rangle
\label{eq_15}
\end{equation}

С учётом ~\ref{eq_9} из условия ~\ref{eq_1} сразу следует $\psi_{11}= 0$

\subsection{Замечание 5}
Обратите внимание, что создание запутанности не может быть обнаружено, глядя на знак первой производной $\mathscr{E}(t)$, если тестовый вектор $\psi \rangle$ не запутан сам.

В самом деле, $\mathscr{E}(t)$ никогда не бывает отрицательным для сепарабельного $\psi \rangle$. Таким образом, оба компонента $\psi_{12}$ и $\psi_{21}$ в ~\ref{eq_15} должны быть отличными от нуля, поскольку в противном случае $\psi \rangle$ становится разделимым.

\subsection{Замечание 6}
Когда $\partial_t \mathscr{E}(0) > 0$ для всех вариантов начального состояния $\rho(0)$ и пробного вектора $| \psi \rangle$, термостат не может перепутать две системы, поскольку $\tilde{\rho}$ остаётся положительным. Обработка случая $\partial_t \mathscr{E}(0) = 0$ требует особой осторожности: для проверки создания запутанности необходимо исследовать производные $\mathscr{E}$ более высокого порядка, возможно, с зависящим от времени $| \psi \rangle$.

Чтобы доказать, что действительно существуют термостаты, для которых $\mathscr{E}(0)=0$ и $\partial_t \mathscr{E}(0)$ отрицательно, сначала сделаем выбор $|a_1\rangle = |b_1\rangle = |+\rangle$ и $|a_2\rangle = |b_2\rangle = |-\rangle$, где $| \pm \rangle$ собственные состояния $\sigma_3$; общий случай рассмотрен ниже.  Для $|\psi \rangle = (|+\rangle \otimes |-\rangle + |-\rangle \otimes |+\rangle ) / \sqrt{2}$
можно найти

\begin{equation}
\partial_t \mathscr{E}(0) = Tr[\mathscr{D}\mathscr{R}]
\label{eq_16}
\end{equation}

,где $\mathscr{D}$ как в ~\ref{eq_7}, а
\begin{equation}
\mathscr{R} = \left(
\begin{array}{cc}
P & Q \\
Q & P \\
\end{array}
\right),
P = \frac{1}{2}\left(
\begin{array}{ccc}
1 & i & 0 \\
-i & 1 & 0 \\
0 & 0 & 0 \\
\end{array}
\right)
\label{eq_17}
\end{equation}

и $Q=diag(-1/2, 1/2, 0)$.  
Хотя $P$ - проектор, $(2Q)^2=diag(1,1,0)$, и, как следствие $\mathscr{R}$ имеет одно отрицательное собственное значение $(1-\sqrt{2})/2$ кратности два. Любой термостат, для которого матрица коэффициентов Косаковского $\mathscr{D}$ имеет поддержку только в отрицательном собственном подпространстве $\mathscr{R}$, будет генерировать отрицательное $\partial_t \mathscr{E}(0)$ и, следовательно, запутать первоначально разделенное состояние $\rho(0)=|+\rangle \langle+| \otimes |+\rangle \langle+|$.

Простой явный пример, в котором это происходит, даётся следующей двухпараметрической матрицей $\mathscr{D}$ с

\begin{equation}
A=C= \left(
\begin{array}{ccc}
1 & -ia & 0 \\
ia & 1 & 0 \\
0 & 0 & 0 \\
\end{array}
\right),
B = \frac{1}{2}\left(
\begin{array}{ccc}
b & 0 & 0 \\
0 & -b & 0 \\
0 & 0 & 0 \\
\end{array}
\right)
\label{eq_18}
\end{equation}

где $a$ и $b$ - действительные постоянные. Положительность $\mathscr{D}$, необходимая для полной положительности марковской динамики подсистемы ~\ref{eq_3}, гарантируется $a^2+b^2 \le 1$. Внутри этого единичного круга область, для которой $\partial_t \mathscr{E}(0)$ отрицательна и характеризуется условием $a+b >1$. Фактически, изменяя начальное состояние $\rho(0)$ и пробный вектор $|\psi\rangle$, можно показать, что сцепление создаётся во всех четырёх участках диска вне вложенного квадрата $|a \pm b| \le 1$ положительно. Обратите внимание, что внутри этого квадрата $\mathscr{D}$, где временная эволюция частично транспонированной матрицы плотности $\tilde{\rho}(t)$, порождённой ~\ref{eq_10}, также полностью положительна: в этом случае запутанность не может быть создана ни при каком выборе начальной состояние $\rho(0)$ и вектора $|\psi \rangle$.
Теперь, когда мы показали, что марковская динамика действительно может запутывать две подсистемы посредством чисто зашумленного механизма: давайте обсудим более подробно условие создания запутанности. Хотя в общем случае базисные векторы $|a_i \rangle, |b_i\rangle$, введённые в ~\ref{eq_9}, не являются собственными состояниями $\sigma_3$, их всегда можно унитарно повернуть к базису $|\pm \rangle$:

\begin{equation}
\begin{gathered}
|a_1\rangle = U|+\rangle \\
|a_2\rangle = U|-\rangle \\
|b_1\rangle = V|+\rangle \\
|b_2\rangle = V|-\rangle \\
\end{gathered}
\label{eq_19}
\end{equation}

Унитарные преобразования U и V индуцируют ортогональные преобразования $\mathscr{U}$ и $\mathscr{V}$ соответственно на матрицах Паули:

\begin{equation}
\begin{gathered}
U^†\sigma_i U = \sum\limits_{j=1}^3 \mathscr{U}_{ij}\sigma_j \\
V^†\sigma_i V = \sum\limits_{j=1}^3 \mathscr{V}_{ij}\sigma_j \\
\end{gathered}
\label{eq_20}
\end{equation}

С этими определениями для общего разделяемого начального состояния ~\ref{eq_9} и произвольного вектора $|\psi\rangle$, такого что $\mathscr{E}(0)=0$, условие $\partial_t \mathscr{E}(0) < 0$ для образования зацепления может быть выражено как следующее математическое ожидание по продукту из $6 \times 6$ матриц:

\begin{equation}
\vec{w}^†\cdot [\psi^† \mathscr{W}^T \tilde{\mathscr{D}} \mathscr{W} \Psi] \cdot \vec{w} <0
\label{eq_21}
\end{equation}

, где $\tilde{\mathscr{D}}$ как в ~\ref{eq_12} (с $H^{(12)}$, установленным в ноль, как объяснялось ранее), а остальные матрицы задаются

\begin{equation}
\mathscr{W}= \left(
\begin{array}{cc}
\mathscr{U} & 0  \\
0 & \mathscr{V}  \\
\end{array}
\right),
\Psi = \left(
\begin{array}{cc}
\psi_{21}1_3 & 0 \\
0 & -\psi_{12}1_3 \\
\end{array}
\right)
\label{eq_22}
\end{equation}

а компоненты 6-вектора $\vec{w}$ матричными элементами Паули:

\begin{equation}
w_i=\langle+|\sigma_i | - \rangle , w_{i+3}=w_i^{*}, i=1,2,3
\label{eq_23}
\end{equation}

Более управляемое условие для проверки образования сцепленности может быть получено, если заметить, что ~\ref{eq_21} квадратично по компонентам $\psi_{12}$ и $\psi_{21}$. Путём соответствующей перестановки выражения в ~\ref{eq_21} можно затем показать, что запутанность возникает, если выполняется следующее неравенство, не зависящее от пробного вектора $|\psi \rangle$:

\begin{equation}
\langle u | A | u \rangle \langle v | C^T | v \rangle <
|\langle u|Re(B)|v\rangle|^2
\label{eq_24}
\end{equation}

3-векторы $|u\rangle$ и $|v\rangle$ не являются полностью произвольными: они содержат информацию о начальном факторизованном состоянии ~\ref{eq_9}, а их компоненты могут быть выражены как

\begin{equation}
u_i=\sum\limits_{j=1}^3 \mathscr{U}_{ij}w_j,
v_i=\sum\limits_{j=1}^3 \mathscr{V}_{ij}w_j^{*},
\label{eq_25}
\end{equation}

Следовательно, данный термостат сможет запутать две подсистемы, развивающиеся с помощью марковской динамики, порождённой ~\ref{eq_3} и характеризуемой матрицей Косаковского ~\ref{eq_7}, если существует начальное состояние $|a_1\rangle \langle a_1 | \otimes |b_1 \rangle \langle b_1 |$ , или, что то же самое, ортогональные преобразования $\mathscr{U}$ и $\mathscr{V}$, для которых выполняется неравенство ~\ref{eq_24}.

Таким образом, условие ~\ref{eq_24} может использоваться для проверки запутывающей способности конкретных марковских временных эволюций. В качестве примера рассмотрим термостат, ведущую к матрице Косаковского ~\ref{eq_7}, для которой $A=B=C$; этот выбор соответствует частному случаю коллективной резонансной флуоресценции
~\cite{b13,18}. Если эрмитова матрица $A$ несимметрична, легко доказать, что существуют начальные состояния вида ~\ref{eq_9} с $|a_1\rangle = |b_1 \rangle$, которые будут запутаны зашумленной динамикой. Действительно, в этом случае условие ~\ref{eq_24} сводится к

\begin{equation}
|\langle u | Im(A) | u \rangle |^2 > 0
\label{eq_26}
\end{equation}

что очевидно выполняется для любого $|u \rangle$ вне собственного нулевого подпространства $Im(A)$. Однако, когда $A$ действительно, ~\ref{eq_26} нарушается, и запутанность не создаётся, поскольку частичное транспонированное состояние $\tilde{\rho}(t)$ развивается во времени с полностью положительной динамикой.

Представленные здесь методы могут быть применены к другим физическим условиям; многообещающей является модель Джейнса-Каммингса для двух двухуровневых систем ~\cite{b13,b19,b20}, где они могут быть использованы для аналитического исследования возможного присутствия «коллапсов» и «возрождений» в поведении запутанности.

\section{Заключение}
Дальнейшее развитие проблематики в ~\cite{b21,b22,b23,b24}

\begin{thebibliography}{3}
\bibitem{benatti}
Benatti, F., Floreanini, R., \& Piani, M. (2003). Environment Induced Entanglement in Markovian Dissipative Dynamics. Physical Review Letters, 91(7). doi:10.1103/physrevlett.91.070402 

\bibitem{b1}	
P. Zanardi, C. Zalka, and L. Faoro, Phys. Rev. A 62, 030301 (2000); P. Zanardi,ibid. 63, 040304 (2001).

\bibitem{b2} 
J. I. Cirac, W. Dür, B. Kraus, and M. Lewenstein, Phys. Rev. Lett. 86, 544 (2001).

\bibitem{b3} 
W. Dür, G. Vidal, J. I. Cirac, N. Linden, and S. Popescu, Phys. Rev. Lett. 87, 137901 (2001).

\bibitem{b4} 
B. Kraus and J. I. Cirac, Phys. Rev. A 63, 062309 (2001).

\bibitem{b5} 
K. Życkowski, P. Horodecki, M. Horodecki, and R. Horodecki, Phys. Rev. A 65, 012101 (2001).

\bibitem{b6} 
D. Braun, Phys. Rev. Lett. 89, 277901 (2002).

\bibitem{b7} 
E. B. Davies, Commun. Math. Phys. 39, 91 (1974); Math. Ann. 219, 147 (1976).

\bibitem{b8} 
V. Gorini, A. Kossakowski, and E. C. G. Sudarshan, J. Math. Phys. (N.Y.) 17, 821 (1976); V. Gorini, A. Frigerio, M. Verri, A. Kossakowski, and E. G. C.
Sudarshan, Rep. Math. Phys. 13, 149 (1978).

\bibitem{b9}
 G. Lindblad, Commun. Math. Phys. 48, 119 (1976).

\bibitem{b10}
 H. Spohn, Rev. Mod. Phys. 52, 569 (1980).

\bibitem{b11}
 R. Alicki and K. Lendi, Quantum Dynamical Semigroups and Applications, Lecture Notes in Physics Vol. 286 (Springer-Verlag, Berlin, 1987).

\bibitem{b12} 
H.-P. Breuer and F. Petruccione, The Theory of Open Quantum Systems (Oxford University Press, Oxford, 2002).

\bibitem{b13}
 R. R. Puri, Mathematical Methods of Quantum Optics (Springer, Berlin, 2001).

\bibitem{b14}
 A. Peres, Phys. Rev. Lett. 77, 1413 (1996).
 
\bibitem{b15}
 M. Horodecki, P. Horodecki, and R. Horodecki, Phys.Lett. A 223, 1 (1996).
 
\bibitem{b18}
 G. S. Agarwal, A. C. Brown, L. M. Narducci, and G. Vetri, Phys. Rev. A 15, 1613 (1977).

\bibitem{b19}
 W. H. Louisell, Quantum Statistical Properties of Radiation (Wiley, New York, 1973).

\bibitem{b20} 
M. O. Scully and M. S. Zubairy, Quantum Optics (Cambridge University Press, Cambridge, 1997).

\bibitem{b21}
 M. S. Kim et al., Phys. Rev. A 65, 040101(R) (2002).

\bibitem{b22}
 S. Schneider and G. J. Milburn, Phys. Rev. A 65, 042107 (2002).
 
\bibitem{b23} 
A. M. Basharov, J. Exp. Theor. Phys. 94, 1070 (2002).

\bibitem{b24} 
L. Jakóbczyk, J. Phys. A 35, 6383 (2002)
	
\end{thebibliography}

\end{document}	